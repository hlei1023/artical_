%%
%% The code below is generated by the tool at http://dl.acm.org/ccs.cfm.
%% Please copy and paste the code instead of the example below.
%%
\begin{CCSXML}
    <ccs2012>
     <concept>
      <concept_id>10010520.10010553.10010562</concept_id>
      <concept_desc>Computer systems organization~Embedded systems</concept_desc>
      <concept_significance>500</concept_significance>
     </concept>
     <concept>
      <concept_id>10010520.10010575.10010755</concept_id>
      <concept_desc>Computer systems organization~Redundancy</concept_desc>
      <concept_significance>300</concept_significance>
     </concept>
     <concept>
      <concept_id>10010520.10010553.10010554</concept_id>
      <concept_desc>Computer systems organization~Robotics</concept_desc>
      <concept_significance>100</concept_significance>
     </concept>
     <concept>
      <concept_id>10003033.10003083.10003095</concept_id>
      <concept_desc>Networks~Network reliability</concept_desc>
      <concept_significance>100</concept_significance>
     </concept>
    </ccs2012>
 \end{CCSXML}
    
\ccsdesc[500]{Computer systems organization~Embedded systems}
\ccsdesc[300]{Computer systems organization~Redundancy}
\ccsdesc{Computer systems organization~Robotics}
\ccsdesc[100]{Networks~Network reliability}

%%
%% Keywords. The author(s) should pick words that accurately describe
%% the work being presented. Separate the keywords with commas.
\keywords{Quantum (micro-) architecture, quantum instruction set architecture, 
superconducting quantum processor}

%% A "teaser" image appears between the author and affiliation
%% information and the body of the document, and typically spans the
%% page.


%%
%% This command processes the author and affiliation and title
%% information and builds the first part of the formatted document.